\documentclass[aps,prb,reprint,groupedaddress,showpacs,amsfonts,amsmath,amssymb,superscriptaddress]{revtex4-1}
\usepackage{graphicx}
\usepackage{bm}
\usepackage{color}
\usepackage[colorlinks,urlcolor=blue,linkcolor=blue,anchorcolor=blue,citecolor=blue,bookmarks]{hyperref}

\begin{document}

\title{Global phase diagram and possible quantum spin liquid in the extended triangular lattice model}

\author{Shi Wang}
\affiliation{National Laboratory of Solid State Microstructures and School of Physics, Nanjing University, Nanjing 210093, China}
\author{Shun-Li Yu}
\email{slyu@nju.edu.cn}
\affiliation{National Laboratory of Solid State Microstructures and School of Physics, Nanjing University, Nanjing 210093, China}
\affiliation{Collaborative Innovation Center of Advanced Microstructures, Nanjing University, Nanjing 210093, China}
\author{Jian-Xin Li}
\email{jxli@nju.edu.cn}
\affiliation{National Laboratory of Solid State Microstructures and School of Physics, Nanjing University, Nanjing 210093, China}
\affiliation{Collaborative Innovation Center of Advanced Microstructures, Nanjing University, Nanjing 210093, China}

\date{\today}

\begin{abstract}
The abstract of this article. Left blank currently.
\end{abstract}

\maketitle

\section{Introduction}
The spin-1/2 Kitaev model on honeycomb lattice, which has both gapped and gapless quantum spin liquid(QSL) ground state (GS) supporting fractionalized excitations, is of central interest in condensed matter physics (and beyond) over the last few years. Because of its theoretical importance and potential application in quantum computing, great efforts have been made to search a solid-state realization of the Kitaev model. As pointed out by Jackeli and Khaliullin, the elementary ingredients for realizing this highly anisotropic spin model is the interplay of the strong relativistic spin-orbit coupling (SOC) and electron interactions. Indeed, the interplay of SOC and electron interactions gives rise to many novel phases, especially for the so called relativistic Mott insulators (RMIs) whose physics may drastically differ from that of Mott insulators with weak SOC (e.g., cuprates). Of particular interest are 4d and 5d transition metal oxides, such as $\alpha$-RuCl$_3$ and iridates A$_2$IrO$_3$ (A=Na, Li).
In 2009, Jackeli and Khaliullin laid out in remarkably precise terms what the elementary ingredients for a succussful Kitaev material serach strategy are. They not only explained the microscopic origin of Kitaev-type bond-directional exchange interactions in certain (edge sharing) 4d$^5$ and 5d$^5$ transition metal compounds, they also proposed the Kitaev-Heisenberg model to describe the physics in iridates of the form A$_2$IrO$_3$ such as Na$_2$IrO$_3$.

\end{document}
