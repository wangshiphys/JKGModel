\documentclass[aps,prb,reprint,groupedaddress,showpacs,amsfonts,amsmath,amssymb,superscriptaddress]{revtex4-1}
\usepackage{graphicx}
\usepackage{bm}
\usepackage{color}
\usepackage[colorlinks,urlcolor=blue,linkcolor=blue,anchorcolor=blue,citecolor=blue,bookmarks]{hyperref}

\begin{document}

\title{Global phase diagram and possible quantum spin liquid in the extended triangular lattice model}

\author{Shi Wang}
\affiliation{National Laboratory of Solid State Microstructures and School of Physics, Nanjing University, Nanjing 210093, China}
\author{Shun-Li Yu}
\email{slyu@nju.edu.cn}
\affiliation{National Laboratory of Solid State Microstructures and School of Physics, Nanjing University, Nanjing 210093, China}
\affiliation{Collaborative Innovation Center of Advanced Microstructures, Nanjing University, Nanjing 210093, China}
\author{Jian-Xin Li}
\email{jxli@nju.edu.cn}
\affiliation{National Laboratory of Solid State Microstructures and School of Physics, Nanjing University, Nanjing 210093, China}
\affiliation{Collaborative Innovation Center of Advanced Microstructures, Nanjing University, Nanjing 210093, China}

\date{\today}

\begin{abstract}
The abstract of this article. Left blank currently.
\end{abstract}

\maketitle

\section{Introduction}
Geometrical frustration, which arises when the lattice geometry gives rise to constraints on the system Hamiltonian that cannot be simultaneously satisfied, plays an important role in various kinds of magnetic systems. In particular, antiferromagnets on the triangular lattice is a typcial example of such geomtrical frustrated spin systems. Anderson proposed that a fully disordered resonating-valence-bond (RVB) state may be the ground state of the antiferromagnetic Heisenberg model on triangular lattice. However, exchange frustation in systems with strongly anisotropic magnetic exchange has been shown  tobe another avenue to explore the QSL state.

\end{document}
