\documentclass[aps,prb,reprint,groupedaddress,showpacs,amsfonts,amsmath,amssymb,superscriptaddress]{revtex4-1}
\usepackage{graphicx}
\usepackage{bm}
\usepackage{color}
\usepackage[colorlinks,urlcolor=blue,linkcolor=blue,anchorcolor=blue,citecolor=blue,bookmarks]{hyperref}

\begin{document}

\title{Global phase diagram and possible quantum spin liquid in the extended triangular lattice model}

\author{Shi Wang}
\affiliation{National Laboratory of Solid State Microstructures and School of Physics, Nanjing University, Nanjing 210093, China}
\author{Shun-Li Yu}
\email{slyu@nju.edu.cn}
\affiliation{National Laboratory of Solid State Microstructures and School of Physics, Nanjing University, Nanjing 210093, China}
\affiliation{Collaborative Innovation Center of Advanced Microstructures, Nanjing University, Nanjing 210093, China}
\author{Jian-Xin Li}
\email{jxli@nju.edu.cn}
\affiliation{National Laboratory of Solid State Microstructures and School of Physics, Nanjing University, Nanjing 210093, China}
\affiliation{Collaborative Innovation Center of Advanced Microstructures, Nanjing University, Nanjing 210093, China}

\date{\today}

\begin{abstract}
\textcolor{red}{Abstract of this article.}

\textcolor{red}{Left blank currently.}
\end{abstract}

\maketitle

\section{Introduction}
Geometric frustration, which arises when the lattice geometry gives rise to constraints that cannot be simultaneously satisfied, plays an important role in various kinds of magnetic systems. In particluar, antiferromagnets on the triangular lattice are typical examples of such geometric frustated spin systems and have attracted numerous interest in condensed matter physics. For nearest-neighbor antiferromagetic Heisenberg model on triangular lattice, Anderson proposed that a fully disordered resonating-valence-bond (RVB) state may be the ground state. However, several theoretical and numerical studies point to a 120$^\circ$ N\'{e}el-ordered ground state. When the next-nearest-neighbor interactions are included which introduce further frustration, the system has a much richer phase diagram incluing a N\'{e}el phase, a phase with spin-wave selection of nontrivial ground states and a phase with incommensurate long-range order. All these studies have revealed that geometric frustated systems show quite different behavior from that of the non-frustated system. \textcolor{red}{$\cdots\cdots$}

Recently however, exchange frustation in systems with strongly anisotropic magnetic exchange has been shown to be another promising approach to explore exotic quantum spin states. Like Geometrical frustation, the effect of exchange frustration is to prevent the formation of long range magnetic order and given raise to a residual ground-state entropy. The spin-1/2 Kitaev model on honeycomb lattice, which has both gapped and gapless quantum spin liquid(QSL) ground state (GS) supporting fractionalized excitations, is of central interest in condensed matter physics (and beyond) over the last few years. Because of its theoretical importance and potential application in quantum computing, great efforts have been made to search a solid-state realization of the Kitaev model. As pointed out by Jackeli and Khaliullin, the elementary ingredients for realizing this highly anisotropic spin model is the interplay of the strong relativistic spin-orbit coupling (SOC) and electron interactions. Indeed, the interplay of SOC and electron interactions gives rise to many novel phases, especially for the so called relativistic Mott insulators (RMIs) whose physics may drastically differ from that of Mott insulators with weak SOC (e.g., cuprates). Of particular interest are 4d and 5d transition metal oxides, such as $\alpha$-RuCl$_3$ and iridates A$_2$IrO$_3$ (A=Na, Li).
In 2009, Jackeli and Khaliullin laid out in remarkably precise terms what the elementary ingredients for a succussful Kitaev material serach strategy are. They not only explained the microscopic origin of Kitaev-type bond-directional exchange interactions in certain (edge sharing) 4d$^5$ and 5d$^5$ transition metal compounds, they also proposed the Kitaev-Heisenberg model to describe the physics in iridates of the form A$_2$IrO$_3$ such as Na$_2$IrO$_3$.
\textcolor{red}{$\cdots\cdots$}

On the experimental side, a variety of triangular magnets such YbMgGaO$_2$, \emph{AReCh}$_2$ (A=alkali or monovalent ions, Re=rare-earth, Ch=O, S,Se) have been synthesized and explored by magnetic susceptibility and heat capacity measurements. The lack of signature for long-range magnetic order and spin freezing down to very low temperature implies their candidacy for quantum spin liquid state. \textcolor{red}{$\cdots\cdots$}

Inspired by previous theoretical and experimental works, we study the J-K-$\Gamma$ model Hamiltonian on the triangular lattice. To the best of our knowledge, no exact solution has been reported so far for the spin-$1/2$ Kitaev and/or $\Gamma$ model on the triangular lattice. Therefore, it remains conceptually interesting to investigate whether the QSL could exist as a possible ground state of the triangular J-K-$\Gamma$ model. \textcolor{red}{$\cdots\cdots$}

In this paper, be combining the classical Monte Carlo simulation and exact diagonalization (ED) calculation, we map out the global phase diagram of the J-K-$\Gamma$ model. \textcolor{red}{$\cdots\cdots$}

The rest of the paper is organized as follows: \textcolor{red}{$\cdots\cdots$}

\end{document}
